\documentclass{article}

% if you need to pass options to natbib, use, e.g.:
% \PassOptionsToPackage{numbers, compress}{natbib}
% before loading nips_2018

% ready for submission
\usepackage[]{nips_2018}

% to compile a preprint version, e.g., for submission to arXiv, add
% add the [preprint] option:
% \usepackage[preprint]{nips_2018}

% to compile a camera-ready version, add the [final] option, e.g.:
% \usepackage[final]{nips_2018}

% to avoid loading the natbib package, add option nonatbib:
% \usepackage[nonatbib]{nips_2018}

\usepackage[utf8]{inputenc} % allow utf-8 input
\usepackage[T1]{fontenc}    % use 8-bit T1 fonts
\usepackage{hyperref}       % hyperlinks
\usepackage{url}            % simple URL typesetting
\usepackage{booktabs}       % professional-quality tables
\usepackage{amsfonts}       % blackboard math symbols
\usepackage{nicefrac}       % compact symbols for 1/2, etc.
\usepackage{microtype}      % microtypography


% The \author macro works with any number of authors. There are
% two
% commands used to separate the names and addresses of multiple
% authors: \And and \AND.
%
% Using \And between authors leaves it to LaTeX to determine where to
% break the lines. Using \AND forces a line break at that point. So,
% if LaTeX puts 3 of 4 authors names on the first line, and the last
% on the second line, try using \AND instead of \And before the third
% author name.
\usepackage{amssymb}
\usepackage{amsmath,amsthm}
\usepackage{float}
\usepackage{url}
\usepackage{setspace}
\usepackage{hyperref}
\usepackage{todonotes}
\usepackage{amsthm} 
\usepackage[ruled,vlined,linesnumbered]{algorithm2e} 
\usepackage{makecell}
\usepackage{upgreek}
\onehalfspacing
\usepackage{pdfpages}
\usepackage{times}
\usepackage{multirow}
\usepackage[toc,page]{appendix}
\usepackage{listings}
\newtheorem{thm}{Theorem}
\newtheorem{lemma}[thm]{Lemma}
\newtheorem{definition}[thm]{Definition}
\newtheorem{observation}[thm]{Observation}
\newtheorem{theorem}[thm]{Theorem}
\newtheorem{claim}[thm]{Claim}
\newtheorem{example}[thm]{Example}
\newtheorem{proposition}[thm]{Proposition}
\newtheorem{corrolary}[thm]{Corrolary}
\usepackage{color,soul}
\usepackage{tikz}
\usetikzlibrary{arrows,shapes.geometric,shapes.arrows}
\usepackage{pgfplots}
\usepackage{multirow}


\DeclareMathOperator{\supp}{support}
\DeclareMathOperator{\support}{support}
\DeclareMathOperator{\Trim}{Trim}
\DeclareMathOperator{\LTrim}{LTrim}
\SetKwFunction{mEqOne}{mEqOne} 
\DeclareMathOperator{\KlmApprox}{KolmogorovApprox}
%\DeclareMathOperator{\OptTrim}{KolmogorovApprox}
\DeclareMathOperator{\OptTrim}{OptTrim}
\title{Kolmogorov Approximation}



\begin{document}
% \nipsfinalcopy is no longer used

\maketitle

\section{Introduction}

Many different approaches to approximation of probability distributions are studied in the literature~\cite{AMCR83,pavlikov2016cvar,PS77}. 
The papers vary in the types random variables involved, how they are represented, and in the criteria used for evaluation of the quality of the approximations. This paper is on approximating discrete distributions represented as explicit probability mass functions with ones that are simpler to store and to manipulate. This is needed, for example, when a discrete distribution is given as a large data-set, obtained, e.g., by sampling, and we want to represent it approximately with a small table.  

The main contribution of this paper is an efficient algorithm for computing the best possible approximation of a given random variable with a random variable whose complexity is not above a prescribed threshold, where the measures of the quality of the approximation and the complexity of the random variable are as specified in the following two paragraphs. 

We measure the quality of an approximation by the distance between the original variable and the approximate one. Specifically, we use the Kolmogorov distance which is one of the most used in statistical practice and literature. Given two random variables $X$ and $X'$ whose cumulative distribution functions (cdfs) are $F_X$ and $F_{X'}$, respectively, the Kolmogorov distance between $X$ and $X'$ is $d_K(X,X')= \sup_t |F_X(t) - F_{X'}(t)|$ (see, e.g.,~\cite{gibbons2011nonparametric}). We say taht $X'$ is a good approximation of $X$ if $d_K(X,X')$ is small.

The complexity of a random variable is measured by the size of its support, the number of values that it can take, $|\support(X)|=|\{x\colon Pr(X=x) \neq 0\}|$. When distributions are maintained as explicit tables, as done in many implementations of statistical software, the size of the support of a variable is proportional to the amount of memory needed to store it and to the complexity of the computations around it. 

In summary, the exact notion of optimality of the approximation targeted in this paper is:
\begin{definition}
	A random variable $X'$ is an optimal $m$-approximation of a random variable $X$ if $|\support(X')| \leq m$ and there is no random variable $X''$ such that $|\support(X'')| \leq m$ and $d_k(X,X'') < d_k(X,X')$.
\end{definition}

The main contribution of the paper is an efficient algorithm that takes $X$ and $m$ as parameters and constructs an optimal $m$-approximation of $X$.

%\begin{theorem} \label{thm:main}
%	Given a random variable $X$ and a number $m$, there is an algorithm with memory and time complexity $O(|\support(X)|^2 \cdot m)$  that computes an optimal $m$-approximation of $X$.
%\end{theorem}

The rest of the paper is organized as follows. In Section~\ref{sec:rel-work} we describe how our work relates to other algorithms and problems studied in the literature. In Section~\ref{sec:alg} we detail the proposed algorithm, analyze its properties, and prove Theorem~\ref{thm:main}. In Section~\ref{sec:exp} we demonstrate how the proposed approach performs on the problem of estimating the probability of hitting deadlines is plans and compare it to alternatives approximation approaches from the literature. We also demonstrate the performance of our approximation algorithm on some randomly generated random variables. The paper is concluded with a discussion in Section~\ref{sec:discussion}.

\section{Related Work}
\label{sec:rel-work}
The problem studied in this paper is related to the theory of Sparse Approximation (aka Sparse Representation) that deals with sparse solutions for systems of linear equations, as follows. 

Given a matrix $D \in \mathbb{R}^{n \times p}$ and a vector $x \in \mathbb{R}^n$, the most studied sparse representation problem is finding the sparsest possible representation $\alpha \in \mathbb{R}^p$ satisfying $x = D\alpha$:
$$
\min_{\alpha \in \mathbb{R}^p} \|\alpha\|_0 \text{ subject to } x = D\alpha.
$$
where $\|\alpha\|_0 = |\{ i : \alpha_i \neq 0, \, i=1,\ldots,p \}|$ is the $\ell_0$ pseudo-norm, counting the number of non-zero coordinates of $\alpha$. This problem is known to be NP-Hard with a reduction to NP-complete subset selection problems.

In these terms, using also the $\ell_\infty$ norm that represents the maximal coordinate and the $\ell_1$ norm that represents the sum of the coordinates, our problem can be phrased as:
$$
\min_{\alpha \in [0,\infty)^p}\|x - D\alpha\|_{\infty} \text{ subject to }  \|\alpha\|_0 = m \text{ and } \|\alpha\|_1=1.
$$
where $D$ is the all-ones triangular matrix (the entry at row $i$ and column $j$ is one if $i\leq j$ and zero otherwise), $x$ is related to $X$ such that the $i$th coordinate of $x$ is $F_X(x_i)$ where $\support(X)=\{x_1 < x_2 < \cdots < x_n\}$ and $\alpha$ is related to $X'$ such that the $i$th coordinate of $\alpha$ is $f_{X'}(x_i)$. The functions $F_X$ and $f_{X'}$ represent, respectively, the cumulative distribution function of $X$ and the mass distribution function of $X'$. This, of course, means that the coordinates of $x$ are assumed to be positive and monotonically increasing and that the last coordinate of $x$ is assumed to be one. We demonstrate an application for this specific sparse representation problem and show that it can be solve in $O(n^2m)$ time and $O(m^2)$ memory.




\section{An Algorithm for Optimal Approximation}\label{sec:alg}

The first useful fact is that it is enough to limit our search to approximations $X'$s such that $\support(X') \subseteq \support(X)$:

\begin{lemma}\label{lem:supContained}
	There is an optimal $m$-approximation $X'$ of $X$ such that $\support(X') \subseteq \support(X)$.
\end{lemma}
\begin{proof}
\end{proof}


Next, note that every random variable $X''$ with support of size at most $m$ that is contained in $\support(X)$ can be described by first setting the (at most $m$) elements of the support of $X''$; then for every such option, determine $X''$ by setting probability values for the elements in the chosen support of $X'$, and setting $0$ for rest of the elements.


\subsection{Step 1: Finding an $X'$ in $\mathbb{X}_S$ that minimizes $d_K(X,X')$}


We first fix a set $S\subseteq \support(X)$ of size at most $m$, and among all the random variables in $\mathbb{X}_S$ find one with a minimal distance from $X$. Denote the elements of $S$ in increasing order by $S=\{x_1<\dots<x_m\}$ and let $x_0=-\infty$, and $x_{m+1}=\infty$. For every $1 <  i \leq m$ let $\hat x_i$ be the maximal element of $\support(X)$ that is smaller than $x_i$.


Next, as the elements of $S$ are also elements of $\support(X)$, we can define the following weight function:

\begin{definition}\label{def:weight} For $0\leq i \leq m$ let
	\[
	w(x_i,x_{i+1})=
	\begin{cases}
	P(x_i < X < x_{i+1}) & \text{if $i=0$ or $i = m$;} \\
	P(x_i < X < x_{i+1})/2 & \text{otherwise.} \\	
	\end{cases}
	\]
\end{definition}

Note that $x_i = -\infty$ for $i=0$ and $x_i=\infty$ for $i=m+1$. Also note that $P(x_i < X < x_{i+1}) = F_X(\hat x_{i+1}) - F_X(x_i)$, a fact that we will use throughout this section.

\begin{definition}
Let $\varepsilon(X,S) = \max\limits_{i=0,\dots,m} w(x_{i}, x_{i+1})$.
\end{definition}



We first show that $\varepsilon(X,S)$ is a lower bound. That is, every random variable in $\mathbb{X}_S$ has a distance at least $\varepsilon(X,S)$. Then, we present a random variable $X'\in \mathbb{X}_S$ with distance $\varepsilon(X,S)$. It then follows that such $X'$ is an optimal $m$-approximation random variable among all random variables in $\mathbb{X}_S$.

The intuition behind choosing these specific weights and $\varepsilon(X,S)$ being a lower bound is as follows.  Since for every $X'\in\mathbb{X}_S$ the probability values of $X'$ for the elements not in $S$ are set to $0$, we have that $F_{X'}(\hat x_{i+1})=F_{X'}(x_i)$. Therefore the distance between $X'$ and $X$ at points $x_i$ and $\hat x_{i+1}$ that we have to take into additional account is increased by $F_X(\hat x_{i+1})-F_X(x_i) = P(x_i < X < x_{i+1})$.

Formally we have the following.

\begin{proposition}\label{prop:minimal}
	If $X'\in\mathbb{X}_S$ then $d_k(X,X') \geq \varepsilon(X,S)$.
\end{proposition}

\begin{proof}
	
	
	By definition, for every $0\leq i\leq m$, $d_k(X,X') \geq \max \{|F_X(\hat x_{i+1}) - F_{X'}(\hat x_{i+1})|,|F_X(x_i) - F_{X'}(x_i)| \}$. Note that $F_{X'}(\hat x_{i+1})=F_{X'}(x_i)$ since the probability values for all the elements not in $S$ are set to $0$.
	
	If $i=0$, that is $x_i=-\infty$, we have that $F_X(x_i)=F_{X'}(x_i)=F_{X'}(\hat x_{i+1})=0$ and therefore $d_k(X,X') \geq |F_X(\hat x_{i+1})| = |F_X(\hat x_{i+1}) - F_{X}(x_i)| =  P(x_i < X < x_{i+1})= w(x_i,x_{i+1})$.
	
	If $i =m$, that is $x_{i+1}=\infty$, we have that $F_X(\hat x_{i+1})=F_{X'}(\hat x_{i+1})=F_{X'}(x_i)=1$. 
	and therefore $d_k(X,X') \geq |1-F_X(\hat x_i)| = |F_X(\hat x_{i+1}) - F_{X}(x_i)| =  P(x_i < X < x_{i+1}) = w(x_i,x_{i+1})$. 
	
	%  we have that $F_X(x_i)=F_{X'}(x_i)=F_{X'}(\hat x_{i+1})=0$ 
	% 
	% 
	% 
	% We break the proof for the closed intervals where each support variable $x_i$ is a proper number, and for the open intervals where a support variable is either $-\infty$ or $\infty$.
	
	Otherwise for every $1\leq i< m$,  we use the fact that $max\{|a|,|b|\} \geq |a-b|/2$ for every $a,b\in\mathbb{R}$, to have $d_k(X,X') \geq 1/2| F_X(\hat x_{i+1}) - F_X(x_i) + F_{X'}(x_i) -F_{X'}(\hat x_{i+1})|$. So $d_k(X,X') \geq 1/2| F_X(\hat x_{i+1}) - F_X(x_i) | =P(x_1 < X < x_2)/2 == w(x_i,x_{i+1})$. 
	
	Therefore since $d_k(X,X') \geq  w(x_i,x_{i+1})$ for every $0\leq i\leq m$, by definition of $\varepsilon(X,S)$ proof follows.
\end{proof}



Next we show a random variable $X'\in\mathbb{X}_S$ with a distance of $\varepsilon(X,S)$ from $X$. Thus $X'$ is an optimal $m$-approximation among the set $\mathbb{X}_S$. We define $X'$ as follows:

\begin{definition}\label{def:construction}
	Let $f_{X'}(x_{i}) = w(x_{i-1},x_i) + w(x_i,x_{i+1}) + f_{X}(x_i)$ for $i=1,\dots,m$ and $f_{X'}(x)=0$ for $x \notin S$.
\end{definition}

We first show that $X'$ is a properly defined random variable:

\begin{lemma}
	$f_{X'}$ is a probability mass function. 
\end{lemma}

\begin{proof}
	From definition $f_{X'}(x_{i})\geq 0$ for every $i$. To see that $\sum_i f_{X'}(x_{i}) =1$, 
	we have $\sum_i f_{X'}(x_{i}) = \sum_i (w(x_{i-1},x_i) + w(x_i,x_{i+1}) + f_{X}(x_i)) = 	
	\sum _{x_i\in S} f_{X}(x_i)) + w(x_0,x_1)+ \sum_{0 < i < m} 2 w(x_i,x_{i+1}) + w(x_m,x_{m+1}) = 
	\sum _{x_i\in S} P(X= x_i) + P(x_0 < X < X_1)+ \sum_{0 < i < m} P(x_i < X < X_{i+1}) +P(x_m < X < X_{m+1}) = 1$ since this sum is the entire cpt of $X$.	
\end{proof}


Note that $X'$ can be constructed in linear time to the size of the cdf of $X$. Intuitively the setting of $X'$ allows to take an "advantage" of distance from $X$ at the elements of $\support(X')$, to avoid the overall increased distance of $X$ from $X'$ at the elements that are not at $\support(X)$ and in which $f_{X'}$ is set to $0$. Formally we have the following.

\begin{lemma}\label{lem:balance}
	Let $x\in\support(X)$ and $0\leq i\leq m$ be such that $x_i\leq x \leq x_{i+1}$ then $-w(x_i,x_{i+1})\leq F_X(x)-F_{X'}(x)\leq  w(x_i,x_{i+1})$.
\end{lemma}

\begin{proof}	
	
	We prove by induction on $0\leq i < m$ .
	
	First see that $F_{X'}(j) = 0$  for every $x_0<j < x_1$ and therefore $F_X(j)-F_{X'}(j) = F_X(j)-0 \leq F_X(\hat x_1)= F_X(\hat x_1)- F_X(x_0) = w(x_0,x_1)$. For $ j = x_1$ we have $F_X(x_1)-F_{X'}(x_1)=  F_X(\hat x_1) +f_{X}(x_1) - (w(x_0,x_1) + w(x_1,x_2) + f_{X}(x_1) = 
	w(x_0,x_1) +f_{X}(x_1) - (w(x_0,x_1) + w(x_1,x_2) + f_{X}(x_1)) = -w(x_1,x_2)$.
	
	Next assume that $F_X(\hat x_i)-F_{X'}(\hat x_i) = w(x_{i-1},x_i)$.
	Then $F_X(x_i)-F_{X'}(x_i) =  F_X(\hat x_i) +f_{X}(x_i) - (w(x_{i-1},x_i) + w(x_i,x_{i+1}) + f_{X}(x_i)) =  w(x_{i-1},x_i) +f_{X}(x_{i}) - (w(x_{i-1},x_i) + w(x_{i},x_{i+1}) + f_{X}(x_{i})) = -w(x_{i},x_{i+1})$.
	
	As before we have that for all $x_i< j< x_{i+1}$, we have $F_X(j)-F_{X'}(j) = F_X(j)-F_{X'}(\hat x_{i+1}) \leq  F_X(\hat x_{i+1})-F_{X'}(\hat x_{i+1})$. Then $F_X(\hat x_{i+1})-F_{X'}(\hat x_{i+1}) = (F_X(x_i)+ P(x_i < x <x_{i+1})) - F_{X'}(x_i) = -w(x_{i},x_{i+1}) + 2w(x_{i},x_{i+1}) = w(x_{i},x_{i+1}) $.
	
	
	Finally for $x_m\leq j \leq x_{m+1}$ we have that $F_{X'}(x_m)=1$ therefore $F_X(x_m)-F_{X'}(x_m) = (1- P(x_m<X < x_{m+1})) - 1 =  P(x_m<X < x_{m+1}) = w(x_m,x_{m+1})$, and for every $x_m<j<x_{m+1}$ we have $F_X(j) - F_{X'}(j) < (1- P(x_m<X < x_{m+1})) - 1 <  - P(x_m<X < x_{m+1}))  = -w(x_m,x_{m+1})$ as required.
\end{proof}

From Lemma \ref{lem:balance}, by the definition of $\varepsilon(X,S)$, we then have:

\begin{corrolary} \label{col:Xprime}
	$d_k(X,X') = \varepsilon(X,S)$.
\end{corrolary}

%\begin{proof}
%	We first see by induction that for every $1\leq i \leq m$ that $F_X(\hat x_i)-F_{X'}(\hat x_i)=w(x_{i-1},x_i)$ and $F_X(x_i)-F_{X'}(x_i) = -w(x_i,x_{i+1})$. 
%	
%	For $i=1$ we have  $F_X(\hat x_1)-F_{X'}(\hat x_1)= F_X(\hat x_1)-0 =  F_X(\hat x_1)- F_X(x_0) = w(x_0,x_1)$.
%	Then $F_X(x_i)-F_{X'}(x_i)=  F_X(\hat x_1) +f_{X}(x_1) - (w(x_0,x_1) + w(x_1,x_2) + f_{X}(x_1)| = 
%	w(x_0,x_1) +f_{X}(x_1) - (w(x_0,x_1) + w(x_1,x_2) + f_{X}(x_1)) = -w(x_1,x_2)$.
%	
%	Assume that $F_X(x_i)-F_{X'}(x_i) = -w(x_i,x_{i+1})$. Then $F_X(\hat x_{i+1})-F_{X'}(\hat x_{i}) = (F_X(x_i)+ P(x_i<X<x_{i+1}))-F_{X'}(x_i) = -w(x_i,x_{i+1})+2w(x_i,x_{i+1})=w(x_i,x_{i+1})$.
%	Therefore, we have $F_X(x_{i+1})-F_{X'}(x_{i+1}) = F_X(\hat x_{i+1}) +f_{X}(x_{i+1}) - (w(x_i,x_{i+1}) + w(x_{i+1},x_{i+2}) + f_{X}(x_{i+1})| = 
%	w(x_i,x_{i+1}) +f_{X}(x_{i+1}) - (w(x_i,x_{i+1}) + w(x_{i+1},x_{i+2}) + f_{X}(x_{i+1})) = -w(x_{i+1},x_{i+2})$.
%	
%	
%	[[DF: read again and see if special instructions are required for $i=0,m$, the last one]]
%	
%	
%	Finally see that for every $x_i<j< x_{i+1}\in \support(X)$, $F_{X'}(j)= F_{X'}(x_i) = F_{X'}(\hat x_{i+1}) $ and therefore  $-w(x_i,x_{i+1}) = F_X(x_i)-F_{X'}(x_i) \leq F_X(j)-F_{X'}(j)\leq F_X(\hat x_{i+1})-F_{X'}(\hat x_{i+1}) = w(x_i,x_{i+1})$. Therefore $|F_X(j)-F_{X'}(j)|\leq w(x_i,x_{i+1})$.
%	
%	Therefore for every $j\in \support(X)$ we have that $|F_X(j)-F_{X'}(j)|\leq max_i(w(x_i,x_{i+1})=\varepsilon(X,S)$. as required.	
%\end{proof}




%\begin{itemize}
%	\item The next lemma states a lower bound on the distance $d_K(X,X')$ when a range of elements is excluded from the support of $X'$.
%\end{itemize}
%
%
%\begin{lemma}\label{lem:geq}
%	For $x_1, x_2 \in \support(X) \cup \{-\infty,\infty\}$ such that $x_1 < x_2$, if $P(x_1 < X' < x_2)=0$  then 
%	$d_k(X,X') \geq P(x_1 < X < x_2)/2$.
%\end{lemma}
%\begin{proof}
%	Let $\hat x=\max \{x \in \support(X) \cap\{ -\infty, \infty\}  \colon x < x_2 \}$. By definition, $d_k(X,X') \geq \max \{|F_X(x_1) - F_{X'}(x_1)|, |F_X(\hat x) - F_{X'}(\hat x)| \}$. From Observation~\ref{obs:ab}, $d_k(X,X') \geq 1/2|F_X(x_1) - F_X(\hat x) + F_{X'}(\hat x) - F_{X'}(x_1)|$. Since it is given that $F_{X'}(\hat x) - F_{X'}(x_1) = P(x_1 < X' < x_2)=0$, $d_k(X,X') \geq 1/2|F_X(x_1) - F_X(\hat x) | =  P(x_1 < X \leq \hat x)/2 = P(x_1 < X < x_2)/2$.
%\end{proof}
%
%
%\begin{itemize}
%	\item The next lemma strengthen the lower bound.
%\end{itemize}
%
%
%\begin{lemma}\label{lem:geq2}
%	For $x_1, x_2 \in \support(X) \cup \{-\infty,\infty\}$ such that $x_1=-\infty$ or  $x_2=\infty$, if $P(x_1 < X' < x_2)=0$  then 
%	$d_k(X,X') \geq P(x_1 < X < x_2)$.
%\end{lemma}
%\begin{proof}
%	Let $\hat x=\max \{x \in \support(X) \cap\{ -\infty, \infty\}  \colon x < x_2 \}$. By definition $d_k(X,X') \geq \max \{|F_X(x_1) - F_{X'}(x_1)|, |F_X(\hat x) - F_{X'}(\hat x)| \}$. If $x_1=-\infty$ then $d_k(X,X') \geq \{|F_X(\hat x) - F_{X'}(\hat x)| \}$ since $F_X(-\infty) = F_{X'}(-\infty) = 0$. Furthermore, $F_{X'}(\hat x) = P(x_1 < X' < x_2)=0$. Therefore $d_k(X,X') \geq F_X(\hat x) = P(x_1 < X \leq \hat x) = P(x_1 < X < x_2)$. 
%	If $x_2=\infty$ then $d_k(X,X') \geq \{|F_X(x_1) - F_{X'}(x_1)| \}$ since $F_X(\hat{x}) = F_{X'}(\hat{x}) = F_X(\infty) = F_{X'}(\infty) = 1$. Furthermore, $F_{X'}(x_1) = 1$ since it is given that $P(x_1 < X' < x_2)=0$. Therefore we get that $d_k(X,X') \geq |F_X(x_1)-1| = |1-F_X(\hat x)-| = P(x_1 < X \leq \hat x) = P(x_1 < X < x_2)$.
%\end{proof}
%
%
%\begin{definition}\label{def:weight} For $x_1,x_2 \in \support(X) \cup \{-\infty,\infty\}$ let
%	\[
%	w(x_1,x_2)=
%	\begin{cases}
%	P(x_1 < X < x_2) & \text{if $x_1=-\infty$ or $x_2 = \infty$;} \\
%	P(x_1 < X < x_2)/2 & \text{otherwise.} \\	
%	\end{cases}
%	\]
%\end{definition} 
%
%
%\begin{definition}\label{def:error} For $S=\{x_1<\dots<x_m\} \subseteq \support(X)$, $x_0=-\infty$, and $x_{m+1}=\infty$, let
%	\[
%	\varepsilon(X,S) = \max\limits_{i=0,\dots,m} w(x_{i}, x_{i+1}).
%	\]
%\end{definition} 
%
%\begin{itemize}
%	\item From here on, until the end of the section, $S$ is fixed.
%\end{itemize}


%
%\begin{proposition}
%	There is no $X'$ such that $\support(X')=S$ and $d_k(X,X') < \varepsilon(X,S)$.
%\end{proposition}
%\begin{proof}
%	Let $i$ be the index that maximizes $w(x_{i}, x_{i+1})$. If $0<i<n-1$ then $d_k(X,X') \geq w(x_{i}, x_{i+1})$ by Lemma~\ref{lem:geq}. If $i=0$ or $i=n+1$ the same follows from Lemma~\ref{lem:geq2}.
%\end{proof}




%
%\begin{lemma}
%	For $i>1$, if $F_{X'}(x_{i})-F_{X}(x_{i}) = w(x_{i}, x_{i+1})$ then $F_{X'}(x_{i+1})-F_{X}(x_{i+1}) = w(x_{i+1}, x_{i+2})$.
%\end{lemma}
%\begin{proof}
%	\begin{align}
%	&F_{X'}(x_{i+1})-F_{X}(x_{i+1}) = \\ \nonumber
%	& = f_{X'}(x_{i+1}) - f_{X}(x_{i+1}) - P(X<x_{i+1}) + P(X'<x_{i+1})  \\ \nonumber
%	& = f_{X'}(x_{i+1}) - f_{X}(x_{i+1}) - F_X(x_{i}) - P(x_{i}< X < x_{i+1}) + F_{X'}(x_{i}) \\ 
%	\label{eq:weight}
%	& = f_{X'}(x_{i+1}) - f_{X}(x_{i+1}) - F_X(x_{i}) - 2w(x_{i},x_{i+1}) + F_{X'}(x_{i})  \\ \label{eq:indHip}
%	& = f_{X'}(x_{i+1}) - f_{X}(x_{i+1}) - 2w(x_{i},x_{i+1}) +w(x_{i}, x_{i+1})  \\ \label{eq:construction}
%	& = w(x_{i},x_{i+1}) +w(x_{i+1},x_{i+2}) - 2w(x_{i},x_{i+1}) +w(x_{i}, x_{i+1}) \\ \nonumber
%	& = w(x_{i+1},x_{i+2}) \nonumber
%	\end{align}
%	
%	By Definition~\ref{def:weight} the probability $P(x_{i-1}< X < x_i) = 2w(x_{i-1},x_{i})$ as in Equation~\eqref{eq:weight}. Equation~\eqref{eq:indHip} is deduced by the induction hypothesis and Equation~\eqref{eq:construction} where
%	$f_{X'}(x_i) - f_{X}(x_i) = w(x_{i-1},x_i) + w(x_i,x_{i+1})$ is true by construction, see Definition\ref{def:construction}.
%\end{proof}

%\begin{lemma}
%	For $i>1$, if $F_{X'}(x_{i-1})-F_{X}(x_{i-1}) = w(x_{i-1}, x_{i})$ then $F_{X'}(x_{i})-F_{X}(x_i) = w(x_i, x_{i+1})$.
%\end{lemma}
%\begin{proof}
%	% TODO switch tdirections to fit the statement of the lemma
%	% TODO \eqref,  aligment
%	\begin{align}
%	&F_X(x_i)-F_{X'}(x_i) = \\
%	&f_{X}(x_i) - f_{X'}(x_i) + P(X<x_i) - P(X'<x_i)  = \\ \nonumber
%	&f_{X}(x_i) - f_{X'}(x_i) + F_X(x_{i-1}) + P(x_{i-1}< X < x_i)-F_{X'}(x_{i-1}) = \\
%	&f_{X}(x_i) - f_{X'}(x_i) + F_X(x_{i-1}) + 2w(x_{i-1},x_{i})-F_{X'}(x_{i-1}) =^* \\
%	&f_{X}(x_i) - f_{X'}(x_i) + 2w(x_{i-1},x_{i}) -w(x_{i-1}, x_{i}) = \\
%	& -w(x_{i-1},x_i) - w(x_i,x_{i+1}) + 2w(x_{i-1},x_{i}) -w(x_{i-1}, x_{i}) =\\ 
%	&- w(x_i,x_{i+1})
%	\end{align}
%	$*$ by induction hypothesis.
%	The probability $P(x_{i-1}< X < x_i) = 2w(x_{i-1},x_{i})$ by Definition~\ref{def:weight}, and
%	$f_{X'}(x_i) - f_{X}(x_i) = w(x_{i-1},x_i) + w(x_i,x_{i+1})$ by construction.
%\end{proof}
%
%\begin{lemma}
%	Base case: $i = 1, F_{X'}(x_{1})-F_{X}(x_{1}) = w(x_{1}, x_{2})$.
%\end{lemma}
%\begin{proof}
%	\begin{align*}
%	F_{X'}(x_{1})-F_{X}(x_{1}) &= \\
%	& = f_{X'}(x_{1}) - f_{X}(x_{1}) - w(x_0, x_1)  \\
%	& = w(x_{0},x_1) + w(x_1,x_{2}) - w(x_0, x_1)  \\
%	& = w(x_1,x_{2})
%	\end{align*}
%\end{proof}
%
%
%


%\begin{proof}
%	Define $X'$ to by $f_{X'}(x_i) = w(x_{i-1},x_i) + w(x_i,x_{i+1}) + f_{X}(x_i)$ for $i=1,\dots,m$ and $f_{X'}(x)=0$ for $x \notin S$.
%	We need to show that $F_X(x_i)-F_{X'}(x_i) = -w(x_i, x_{i+1})$. Assume this is true for every $j<i$, the induction hypothesis hereby: $F_X(x_{i-1})-F_{X'}(x_{i-1}) = -w(x_{i-1}, x_{i})$.
%	\begin{align*}
%		&F_X(x_i)-F_{X'}(x_i) = \\
%		&f_{X}(x_i) - f_{X'}(x_i) + P(X<x_i) - P(X'<x_i)  = \\
%		&f_{X}(x_i) - f_{X'}(x_i) + F_X(x_{i-1}) + P(x_{i-1}< X < x_i)-F_{X'}(x_{i-1}) = \\
%		&f_{X}(x_i) - f_{X'}(x_i) + F_X(x_{i-1}) + 2w(x_{i-1},x_{i})-F_{X'}(x_{i-1}) =^* \\
%		&f_{X}(x_i) - f_{X'}(x_i) + 2w(x_{i-1},x_{i}) -w(x_{i-1}, x_{i}) = \\
%		& -w(x_{i-1},x_i) - w(x_i,x_{i+1}) + 2w(x_{i-1},x_{i}) -w(x_{i-1}, x_{i}) =\\ 
%		&- w(x_i,x_{i+1})
%	\end{align*}
%	$*$ by induction hypothesis.
%	The probability $P(x_{i-1}< X < x_i) = 2w(x_{i-1},x_{i})$ by definition~\ref{def:weight}, and
%	$f_{X'}(x_i) - f_{X}(x_i) = w(x_{i-1},x_i) + w(x_i,x_{i+1})$ by construction.
%	
%\end{proof}



%\begin{proposition}
%	For any random variable $X$ and an ordered set $S=\{x_1<\dots<x_m\} \subset \support(X)$ there is no random variable $X'$ such that $\support(X')=S$ and $d_k(X,X') < \max\limits_{i=0,\dots,m} w(x_{i}, x_{i+1})$ where, to simplify notations, we assume that $x_0=-\infty$ and $x_{m+1}=\infty$.
%\end{proposition}
%
%\begin{proof}
%	Let $i$ be the index that maximizes $w(x_{i}, x_{i+1})$. If $0<i<n-1$ then $d_k(X,X') \geq w(x_{i}, x_{i+1})$ by Lemma~\ref{lem:geq}. If $i=0$ or $i=n+1$ the same follows from Lemma~\ref{lem:geq2}.
%\end{proof}
%
%\begin{itemize}
%	\item The next lemma shows that there is a n not better approximation. 
%\end{itemize}
%
%
%
%\begin{proposition}
%	For any random variable $X$ and an ordered set $S=\{x_1<\dots<x_m\} \subset \support(X)$ there is a random variable $X'$ such that $\support(X')=S$ and $d_k(X,X') = \max\limits_{i=0,\dots,m} w(x_{i}, x_{i+1})$ where, to simplify notations, we assume that $x_0=-\infty$ and $x_{m+1}=\infty$.
%\end{proposition}
%\begin{proof}
%Define $X'$ to by $f_{X'}(x_i) = w(x_{i-1},x_i) + w(x_i,x_{i+1}) + f_{X}(x_i)$ for $i=1,\dots,m$ and $f_{X'}(x)=0$ for $x \notin S$.
%\end{proof}
\subsection{Step 2: Finding an $S$ that minimizes $\varepsilon(X,S)$}
Chakravarty, Orlin, and Rothblum~\cite{chakravarty1982partitioning} proposed a polynomial-time method that, given a certain objective functions (additive), finds an optimal consecutive partition. Their method involves the construction of a graph such that the (consecutive) set partitioning problem is reduced to the problem of finding the shortest path in that graph.

The $\KlmApprox$ algorithm (Algorithm~\ref{alg:optapprox}) starts by constructing a directed weighted graph $G$ similar to the method of Chakravarty, Orlin, and Rothblum~\cite{chakravarty1982partitioning}. The nodes $V$ consist of the support of $X$ together with an extra two nodes, $-\infty$ and $\infty$ for technical reasons, whereas the edges $E$ connect every pair of nodes in one direction (lines 1-2). The weight $w$ of each edge $e=(x,y)\in E$ is determined by one of two cases as in Definition~\ref{def:weight}. The first is where nodes $x$ or $y$ are the source or target nodes respectively. In this case, the weight is the probability of $X$ to get a value between $x$ and $y$, non inclusive, i.e., $w(e)=Pr(x<X<y)$. The second case is where $x$ and $y$ are not a source or target nodes, here the weight is the probability of $X$ to get a value between $x$ and $y$, non inclusive, divided by two i.e., $w(e)=Pr(x<X<y)/2$. The values taken are non inclusive, since we are interested only in the error value. 
The source node of the shortest path problem at hand corresponds to the $-\infty$ node added to $G$ in the construction phase, and the target node is the extra node $\infty$.
The set of all solution paths in $G$, i.e., those starting at $-\infty$ and ending in $\infty$ with at most $m$ edges, is called $paths(G, -\infty, \infty)$. The goal is to find the path $l$ in $paths(G, -\infty, \infty)$ with the lightest bottleneck (line 3). This can be achieved by using the $Bellman-Ford$ algorithm with two tweaks. The first is to iterate the graph $G$ in order to find only paths with length of at most $m$ edges. The second is to find the lightest bottleneck as opposed to the traditional objective of finding the shortest path. This is performed by modifying the manner of ``relaxation'' to $bottleneck(x) = min[max(bottleneck(v),w(e))]$, done also in~\cite{shufan2011two}. Consequently, we find the lightest maximal edge in a path of length $\leq m$, which represents the minimal error, $\varepsilon(X,S)$, defined in Definition~\ref{def:error} where the nodes in path $l$ represent the elements in set $S$. The approximated random variable $X'$ is then derived from the resulting path $l$ (lines 4-5). Every node $x\in l$ represent a value in the new calculated random variable $X'$, we than iterate the path $l$ to find the probability of the event $f_{X'}(x)$ as described in Definition~\ref{def:construction}. For every edge $(x_i,x_j)\in l$ we determine: if $(x_i,x_j)$ is the first edge in the path $ l$ (i.e.  $x_i=-\infty$), then node $x_j$ gets the full weight $w(x_i,x_j)$ and it's own weight in $X$ such that $f_{X'}(x_j) = f_{X}(x_j) +  w(x_i,x_j)$. If $(x_i,x_j)$ in not the first nor the last edge in path $l$ then we divide it's weight between nodes $x_i$ and $x_j$ in addition to their own original weight in $X$ and the probability that already accumulated. If $(x_i,x_j)$ is the last edge in the path $ l$ (i.e.  $i=\infty$) then node $_i$ gets the full weight $w(x_i,x_j)$ in addition to what was already accumulated such that $f_{X'}(x_j) = f_{X'}(x_j) +  w(x_i,x_j)$.



\begin{algorithm}\label{alg:optapprox}
	\DontPrintSemicolon
	\SetKwFunction{Convolv}{Conv}
	\SetKwFunction{getPartition}{getPartition}
	\SetKwFunction{bellmanFordMinMaxM}{bellmanFordMinMaxM}
	$S = \support(X)\cup \{\infty,-\infty\}$\;
	$G=(V,E)=(S, \{ (x,y)  \colon  x<y \})$  \;
	$(x_0,\dots,x_{m+1}) = l \in \operatorname{argmin}\limits_{l \in paths(G,-\infty,\infty),|l|\leq m}  \max \{ w(e)\colon e \in l  \}$  \;
	\For{$0<i<m+1$ }{
		$f_{X'}(x_{i}) = w(x_{i-1},x_i) + w(x_i,x_{i+1}) + f_{X}(x_i)$
	}	
	\Return $X'$\;
	
	\caption{$\KlmApprox (X, m)$}  
	\label{alg:sequence}
\end{algorithm}


%\begin{algorithm}\label{alg:optapprox}
%	\DontPrintSemicolon
%	\SetKwFunction{Convolv}{Conv}
%	\SetKwFunction{getPartition}{getPartition}
%	\SetKwFunction{bellmanFordMinMaxM}{bellmanFordMinMaxM}
%	$S = \support(X)\cup \{\infty,-\infty\}$\;
%	$G=(V,E)=(S, \{e= (x,y) \in S^2 \colon  x<y \})$ \;
%	
%	\ForEach{$e=(x,y) \in  E $ }{
%		\If {$i=\infty \text{ OR } j=-\infty$}{
%			$w(e) = Pr(i<X<j)$
%		}
%		\Else{$w(e) = Pr(i<X<j)/2$}
%	}
%	
%	/* The following can be obtained, e.g., using the Bellman-Ford algorithm */\;
%	$l^*= \operatorname{argmin}\limits_{l \in paths(G, -\infty, \infty,|l|\leq m}  \max \{ w(e)\colon e \in l  \}$ \;
%	
%	\ForEach{$e=(i,j) \in  l^* $ }{
%		\If {$i=-\infty$}{
%			$f_{X'}(j) = f_{X}(j) + Pr(i\leq X<j)$
%		}
%		\ElseIf{$j==\infty$}{
%			$f_{X'}(i) = f_{X'}(i) + Pr(i\leq X<j)$
%		}
%		\Else{
%			$f_{X'}(i) = f_{X'}(i) + Pr(i\leq X<j)/2$\;
%			$f_{X'}(j) = f_{X}(j) + Pr(i\leq X<j)/2$\;
%		}
%		
%	}	
%	\Return $X'$\;
%	
%	\caption{$\KlmApprox (X, m)$}  
%	\label{alg:sequence}
%\end{algorithm}

\begin{theorem}\label{the:algo}
	$\KlmApprox(X,m)$ is an $m$-optimal-approximation of $X$.
\end{theorem}
\begin{proof}
If we consider the vertexes $S= l \setminus \{-\infty,\infty\}$ for a path $l \in paths(G,-\infty,\infty)$ we have that $\max \{ w(e)\colon e \in l  \} = \varepsilon(X,S)$. Therefore, line 3 of the algorithm essentially computes a set $S \in \operatorname{argmin}_{S \subseteq \support(X), |S|\leq m} \varepsilon(X,S)$. By Corollary~\ref{col:Xprime}, the variable $X'$ constructed in lines 4 and 5 satisfies $d_K(X,X') = \varepsilon(X,S)$ and by the minimality of $S$ and by Proposition~\ref{prop:minimal}, it is an optimal approximation.
\end{proof}


\begin{theorem}\label{the:complexity}
	The $\KlmApprox(X,m)$ algorithm runs in time $O(mn^2)$, using $O(n^2)$ memory where $n=|\support(X)|$.
\end{theorem}

\begin{proof}
	Constructing the graph $G$ takes $O(n^2)$. The number of edges is $O(E)\approx O(n^2)$ and for every edge the weight is at most the sum of all probabilities between the source node $-\infty$ and the target node $\infty$, which can be done efficiently by aggregating the weights of already calculated edges. 
	The construction is also the only stage that requires memory allocation, specifically $O(E+V)=O(n^2)$.
	Finding the shortest path takes $O(m(E+V))\approx O(mn^2)$. 
	
	[[GW: put a reference to the work of the fellows from the Technion to avoid some of this?]]
	
	Since $G$ is DAG (directed acyclic graph) finding shortest path takes $O(E+V)$. We only need to find paths of length $\leq m$, which takes $O(m(E+V))$.
	Deriving the new random variable $X'$ from the computed path $l$ takes $O(mn)$. For every node in $l$ (at most $m$ nodes), calculating the probability $P(s<X<\infty)$ takes at most $n$ steps. 
	To conclude, the worst case run-time complexity is $O(n^2+mn^2+mn)=O(mn^2)$ and memory complexity is $O(E+V)=O(n^2)$.
\end{proof}



\section{A case study and experimental results}\label{sec:exp}
The case study examined in our experiments is the problem of task trees with deadlines~\cite{cohen2015estimating,CohenGW18}. Hierarchical planning is a well-established field in AI~\cite{thomas1988hierarchical,erol1994htn,erol1996complexity}, and is still relevant nowadays~\cite{alford2016hierarchical,xiao2017hierarchical}. A hierarchical plan is a method for representing problems of automated planning in which the dependency among tasks can be given in the form of networks, here we focus on hierarchical plans represented by task trees. The leaves in a task tree are \emph{primitive} actions (or tasks), and the internal nodes are either \emph{sequence} or \emph{parallel} actions. The plans we deal with are of stochastic nature, where the duration of a primitive action is given by a random variable. 


A sequence node denotes a series of tasks that should be performed consecutively, whereas a parallel node denotes a set of tasks that begin at the same time. A \emph{valid} plan is one that is fulfilled before some given \emph{deadline}, i.e., its \emph{makespan} is less than or equal to the deadline. The objective in this context is to compute the probability that a given plan is valid, or more formally computing $P(X<T)$, where $X$ is a random variable representing the makespan of the plan and $T$ is the deadline. As said above, resource consumption (task duration) is uncertain, and described as probability distributions in the leaf nodes.
We assume that the distributions are independent but {\em not} necessarily identically distributed and that the random variables are discrete and have a finite support. 

The problem of finding the probability that a task tree satisfies a deadline is known to be NP-hard. In fact, even the problem of summing a set of random variables is NP-hard~\cite{mohring2001scheduling}. This is an example of an explicitly given random variable that we need to estimate deadline meeting probabilities for.

In the first experiment we focus on is the problem of task trees with deadlines, and consider three types of task trees. The first type includes logistic problems of transporting packages by trucks and airplanes (from IPC2 http://ipc.icaps-conference.org/). Hierarchical plans of those logistic problems were generated by the JSHOP2 planner \cite{nau2003shop2} (see example problem, Figure~\ref{fig:logistics}, one parallel node with all descendant task nodes being in sequence). 
The second type consists of task trees used as execution plans for the ROBIL team entry in the DARPA robotics challenge (DRC simulation phase), and the third type is of linear plans (sequential task trees).
The primitive tasks in all the trees are modeled as discrete random variables with support of size $M$ obtained by discretization of uniform distributions over various intervals. The number of tasks in a tree is denoted by $N$. 


We implemented the approximation algorithm for solving the deadline problem with four different methods of approximation. The first two are for achieving a one-sided Kolmogorov approximation -- the $\OptTrim$~\cite{CohenGW18} and the $\Trim$~\cite{cohen2015estimating} operators, and the third is a simple sampling scheme. We used those methods as a comparison to the Kolmogorov approximation with the suggested $\KlmApprox$ algorithm. 
The parameter $m$ of $\OptTrim$ and $\KlmApprox$ corresponds to the inverse of $\varepsilon$ given to the $\Trim$ operator. Note that in order to obtain some error $\varepsilon$, one must take into consideration the size of the task tree $N$, therefore, $m/N=1/(\varepsilon\cdot N)$. We ran also an exact computation as a reference to the approximated one in order to calculate the error. The experiments conducted with the following operators and their parameters: $\KlmApprox$ operator with $m=10\cdot N$, the $\OptTrim$ operator with $m=10\cdot N$, the $\Trim$ as operator with $\varepsilon={0.1/N}$, and two simple simulations, with a different samples number $s=10^4$ and $s=10^6$. 
\begin{table}[th]
	\scriptsize
	\centering
	\renewcommand{\arraystretch}{1.3}
	\begin{tabular}{|c|c|c|c|c|c|c|}
		\hline
		\multirow{2}{*}{Task Tree} & \multirow{2}{*}{$M$} & {$\KlmApprox$} & {$\OptTrim$} & {$\Trim$} & \multicolumn{2}{c|}{Sampling} \\ \cline{3-7} 
		&	& $m/N{=}10$ & $m/N{=}10$ & $\varepsilon\cdot N{=}0.1$ & $s{=}10^{4}$& $s{=}10^{6}$ \\ \hline
		\hline
		
		%N=34
		
		\multirow{2}{*}{Logistics} & 2& 0 & 0 &  0.0019 &  0.007 & 0.0009  \\ \Xcline{2-7}{1pt}
		{\tiny $(N=34)$}& 4& 0.0024 & 0.0046&  0.0068  &   0.0057 & 0.0005 \\\Xhline{1pt}
		
		%N=45
		\multirow{2}{*}{Logistics}  & {2} & 0.0002 & 0.0005 &  0.002 &  0.015& 0.001
		\\ \Xcline{2-7}{1pt} 
		{\tiny $(N{=}45)$} & {4} & 0 & 0.003 & 0.004 & 0.008 & 0.0006  
		\\\Xhline{1pt}
		
		%N=47
		\multirow{2}{*}{DRC-Drive}  
		&2	& 0 & 0.004&  0.009  & 0.0072 & 0.0009  
		\\ \Xcline{2-7}{1pt}
		
		{\tiny $(N{=}47)$}& {4}& 0.001 & 0.008&  0.019   & 0.0075  & 0.0011 
		\\  \Xhline{1pt}
		
		
		%N=10
		\multirow{3}{*}{Sequential}  & {2} & 0.0093 & 0.015 &  0.024 & 0 & 0 \\ \Xcline{2-7}{1pt}  
		& {4} & 0.008 & 0.024 &  0.04 & 0.008 & 0.0016 \\ \Xcline{2-7}{1pt} 
		{\tiny $(N{=}10)$} & {10} & 0 & 0.028  &  0.06  & 0.0117  & 0.001 \\\Xhline{1pt}
		
		
		
	\end{tabular}
	\caption{Comparison of estimated errors with respect to the reference exact computation on various task trees.}
	\label{tab:errors}
	
\end{table}

Table~\ref{tab:errors} shows the results of the main experiment. The quality of the solutions provided by using the $\OptTrim$ operator are better (lower errors) than those provided by the $\Trim$ operator, following the optimality guarantees, but is interesting to see that the quality gaps happen in practice in each of the examined task trees. However, in some of the task trees the sampling method produced better results than the approximation algorithm with $\OptTrim$. Nevertheless, the approximation algorithm comes with an inherent advantage of providing an exact quality guarantees, as opposed to the probabilistic guarantees provided by sampling.

In order to better understand the quality gaps in practice between $\KlmApprox$, $\OptTrim$, and $\Trim$, we investigate their relative errors when applied on single random variables with support size $n = 100$, and different support sizes of the resulting random variable approximation ($m$). In each instance of this experiment, a random variable is randomly generated by choosing the probabilities of each element in the support from a uniform distribution and then normalizing these probabilities so that they sum to one.

Figure~\ref{fig:error} present the error produced by the above methods. The depicted results are averages over several instances (50 instances) of random variables. The curves in the figure show the average error of $\OptTrim$ and $\Trim$ operators with comparison to the average error of the optimal approximation provided by $\KlmApprox$ as a function of $m$.  

According to the depicted results it is evident that increasing the support size of the approximation $m$ reduces the error, as expected, in all three methods. However, errors produced by the $\KlmApprox$ are significantly smaller, safe to say, a half of the error produced by $\OptTrim$ and $\Trim$, it is clear both in the table (the relative error is mostly above 1) and in the graph.




We also examined how our algorithm compares to linear programing as described and discussed, for example, in~\cite{pavlikov2016cvar}. We ran an experiment to compare the run-time between the $\KlmApprox$ algorithm with the run-time of a state-of-art implementation of linear programing. We used the ``Minimize'' function of Wolfram Mathematica and fed it with the equations $\min_{\alpha \in \mathbb{R}^n} \| x - \alpha\|_\infty$ subject to $\|\alpha\|_0 \leq m$ and $\| \alpha \|_1 =1$ . The run-time comparison results were clear and persuasive, for a random variable with support size $n=10$ and $m=5$, the LP algorithm run-time was 850 seconds, where the $\KlmApprox$ algorithm run-time was less than a tenth of a second. For $n=100$ and $m=5$, the $\KlmApprox$ algorithm run-time was 0.14 seconds and the LP algorithm took more than a day. Due to these timing results of the LP algorithm we did not proceed to examine it any further.
Since it is not trivial to formally analyze the run-time of the LP algorithm, we conclude by the reported experiment that in this case the LP algorithm might not be as efficient as $\KlmApprox$ algorithm whose complexity is proven to be polynomial in Theorem~\ref{the:complexity}.

\section{Discussion}\label{sec:discussion}


\bibliography{library,Trim_Optimum}{}
\bibliographystyle{abbrv}

\end{document}

